\documentclass{notes}

  \title{Trees}
  \author{ian.mcloughlin@gmit.ie}
  \date{\today}

\begin{document}

  \section*{Definitions}
  \begin{description}
    \item[Tree:] connected graph with no cycles.
    \item[Rooted:] if one node identified as root.
    \item[$m$-ary:] if every parent has $m$ children.
    \item[Height:] maximum length of path to a leaf.
    \item[Leaf:] is a node with no children.
  \end{description}
  
  
  \section*{Examples}

    \begin{center}
      \begin{forest}
        for tree={circle, draw, fill}
        [
          [
            [
              []
              []
            ]
            [
              []
              []
            ]
          ]
          []
        ]
      \end{forest}
    \end{center}

  Binary ($2$-ary) tree of height 3.

  \vspace{6mm}

  \begin{center}
    \begin{forest}
      for tree={circle, draw, fill}
      [
        [
          []
          []
          []
        ]
        [
          []
          []
          []
        ]
        [
          []
          []
          []
        ]
      ]
    \end{forest}
  \end{center}

Ternary ($3$-ary) tree of height 2.

\vspace{6mm}

\begin{center}
  \begin{forest}
    for tree={circle, draw, fill}
    [
      []
      []
      []
      []
    ]
  \end{forest}
\end{center}

$4$-ary tree of height 1.

  \section*{Theorem}
    $m$-ary tree with $l$ leaves has height at least $\log_m l$.

  \section*{Log}

    \[ b^a = c \Leftrightarrow \log_b c = a \]

    \[ 10^2 = 100 \Leftrightarrow \log_{10} 100 = 2 \]

  \section*{Rationale}
    Maximum leaves is $l \leq m^h$.

  \section*{Proof of theorem}

    \[h \geq log_m l \  \Leftrightarrow \  m^h \geq m^{log_m l} \Leftrightarrow m^h \geq l \]

    \vspace{2mm}

    By induction on $h$:

    \begin{description}
      \item[$h = 0$:] $m^0 = 1$, $l = 1$, $1 \geq 1$.
      \item[$h = n$:] Assume true for $h = n - 1$. Removing root gives $m$ trees with maximum $m^{n-1}$ leaves each. In total, we get maximimum of $m(m^{n-1}) = m^n$ leaves.
    \end{description}

    True for $h = 0$. Therefore true for $h = 0 + 1 = 1$. Therefore true for $h = 1 + 1 = 2$. Therefore true for $h = 2 + 1 = 3$. And so on.

    \section*{Example}

    \begin{center}
      \begin{forest}
        for tree={circle, draw, fill}
        [
          [,for tree={fill=red}
            [
              []
              []
            ]
            [
              []
              []
            ]
          ]
          [,for tree={fill=green}
            [
              []
              []
            ]
            [
              []
              []
            ]
          ]
        ]
      \end{forest}
    \end{center}

  %\bibliography{bibliography}
\end{document}