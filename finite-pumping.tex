\documentclass[a4paper, hidelinks, twocolumn, 9pt]{article}
  % Change page geometry.
  \usepackage[a4paper, right=20mm, left=20mm, top=20mm, bottom=20mm]{geometry} 
  % Enables the use of colour.
  \usepackage{xcolor}
  % Syntax high-lighting for code. Requires Python's pygments.
  \usepackage{minted}
  % Enables the use of umlauts and other accents.
  \usepackage[utf8]{inputenc}
  % Diagrams.
  \usepackage{tikz}
  % Settings for captions, such as sideways captions.
  \usepackage{caption}
  % Symbols for units, like degrees and ohms.
  \usepackage{gensymb}
  % Latin modern fonts - better looking than the defaults.
  \usepackage{lmodern}
  % Allows for columns spanning multiple rows in tables.
  \usepackage{multirow}
  % Better looking tables, including nicer borders.
  \usepackage{booktabs}
  % More math symbols.
  \usepackage{amssymb}
  % More math fonts, like \mathbb.
  \usepackage{amsfonts}
  % More math layouts, equation arrays, etc.
  \usepackage{amsmath}
  % More theorem environments.
  \usepackage{amsthm}
  % More column formats for tables. 
  \usepackage{array}
  % Adjust the sizes of box environments.
  \usepackage{adjustbox}
  % Better looking single quotes in verbatim and minted environments.
  \usepackage{upquote}
  % Better blank space decisions.
  \usepackage{xspace}
  % Better looking tikz trees.
  \usepackage{forest}
  % URLs.
  \usepackage{hyperref}
  % Plotting.
  \usepackage{pgfplots}
  % Changing headers and footers.
  \usepackage{fancyhdr}
  % Calculates the number of pages.
  \usepackage{lastpage}
  % Styling the abstract.
  \usepackage{abstract}
  % Titling
  \usepackage{titling}
  
  % Various tikz libraries.
  % For drawing mind maps.
  \usetikzlibrary{mindmap}
  % For adding shadows.
  \usetikzlibrary{shadows}
  % Extra arrows tips.
  \usetikzlibrary{arrows.meta}
  % Old arrows.
  \usetikzlibrary{arrows}
  % Automata.
  \usetikzlibrary{automata}
  % For more positioning options.
  \usetikzlibrary{positioning}
  % Creating chains of nodes on a line.
  \usetikzlibrary{chains}
  % Fitting node to contain set of coordinates.
  \usetikzlibrary{fit}
  % Extra shapes for drawing.
  \usetikzlibrary{shapes}
  % For markings on paths.
  \usetikzlibrary{decorations.markings}
  % For advanced calculations.
  \usetikzlibrary{calc}
  
  % GMIT colours.
  \definecolor{gmitblue}{RGB}{20,134,225}
  \definecolor{gmitred}{RGB}{220,20,60}
  \definecolor{gmitgrey}{RGB}{67,67,67}

  % Blank symbols.
  \newcommand{\blm}{\sqcup}
  \newcommand{\bl}{\(\blm\)}
  
  % Abstract
  \renewcommand{\abstractname}{}
  \renewcommand{\absnamepos}{empty}

  % Environments
  \newtheorem*{remark}{Remark}
  \newtheorem*{definition}{Definition}

  % Tables
  \newcolumntype{x}[1]{>{\centering\arraybackslash\hspace{0pt}}p{#1}}

  % Bibliography
  \renewcommand{\refname}{\selectfont\normalsize References} 

  
  \pgfplotsset{compat=1.16}
  
  \pagestyle{fancy}
  \fancyhf{}
  \renewcommand{\headrulewidth}{0pt}
  \rhead{\url{ian.mcloughlin@gmit.ie}}
  
  \lhead{\today}
 
  \setlength{\droptitle}{-40pt}
  %%begin novalidate
  \posttitle{\par\end{center}\vspace{-50pt}}
  %%end novalidate
  \setlength{\columnsep}{20mm}
  \setlength{\columnseprule}{0.1mm}
  \setlength{\parindent}{0pt} 
 
  % Title, etc.
  \title{Pumping Lemma for Finite Automata}
  \author{}
  \date{}

\begin{document}
  
  \maketitle
  \thispagestyle{fancy}

  
  \section*{Theorem}
    Let \(L\) be an infinite regular language.
    Let \(p\) be the number of states in a deterministic finite automaton that recognises \(L\).
    Then any string \(s\) in \(L\) of length at least \(p = |s|\) can be broken into three substrings \(s=xyz\) such that:
    \begin{itemize}
      \item \(|y| > 0\) ,
      \item \(|xy| \leq p\), and
      \item \(xy^iz \in L\) for all \(i \in \mathbb{N}_0\).
    \end{itemize}
  
    \section*{Rationale}
      Once we read \(p\) characters from \(s\), we must have visited some state twice\footnote{At least one state at least twice.}.
      Suppose \(q\) is a state we visit twice, and call \(y\) the substring of \(s\) that we read between the two visits.

      When we can delete \(y\) from \(s\) and the automaton must accept this new string also.
      Likewise we can repeat \(y\) any number of times to create a new string that must also be accepted.

      Note the automaton essentially forgets the path it took to a given state -- once it arrives at a given state it can't remember how it got there.

    \section*{Example}

    \section*{Non-regular example}
      \[L = \{0^i1^i | i \in \mathbb{N}_0\}\]
      \[s = 0^p1^p = xyz \]
      \[|xy| \leq p \Rightarrow y = 0^n, n \in \mathbb{N}\]
      \[\Rightarrow 0^{p-n}0^{2n}1^n \in L \]
      That's a contradiction.

    
  \bibliographystyle{plain}
  \bibliography{bibliography}
\end{document}