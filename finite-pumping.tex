\documentclass{notes}

  \title{Pumping Lemma}
  \author{ian.mcloughlin@gmit.ie}
  \date{\today}

\begin{document}

  

  

  
  \section*{Theorem}
    Let \(L\) be an infinite regular language.
    Let \(p\) be the number of states in a deterministic finite automaton that recognises \(L\).
    Then any string \(s\) in \(L\) of length at least \(p = |s|\) can be broken into three substrings \(s=xyz\) such that:
    \begin{itemize}
      \item \(|y| > 0\) ,
      \item \(|xy| \leq p\), and
      \item \(xy^iz \in L\) for all \(i \in \mathbb{N}_0\).
    \end{itemize}
  
    \section*{Rationale}
      Once we read \(p\) characters from \(s\), we must have visited some state twice\footnote{At least one state at least twice.}.
      Suppose \(q\) is a state we visit twice, and call \(y\) the substring of \(s\) that we read between the two visits.

      When we can delete \(y\) from \(s\) and the automaton must accept this new string also.
      Likewise we can repeat \(y\) any number of times to create a new string that must also be accepted.

      Note the automaton essentially forgets the path it took to a given state -- once it arrives at a given state it can't remember how it got there.

    \section*{Example}

    \section*{Non-regular example}
      \[L = \{0^i1^i | i \in \mathbb{N}_0\}\]
      \[s = 0^p1^p = xyz \]
      \[|xy| \leq p \Rightarrow y = 0^n, n \in \mathbb{N}\]
      \[\Rightarrow 0^{p-n}0^{2n}1^n \in L \]
      That's a contradiction.

  \bibliography{bibliography}
\end{document}