\documentclass[a4paper, hidelinks, twocolumn, 9pt]{article}
      \usepackage[a4paper, right=20mm, left=20mm, top=20mm, bottom=20mm]{geometry}  % Changing headers and footers.
  % Enables the use of colour.
  \usepackage{xcolor}
  % Syntax high-lighting for code. Requires Python's pygments.
  \usepackage{minted}
  % Enables the use of umlauts and other accents.
  \usepackage[utf8]{inputenc}
  % Diagrams.
  \usepackage{tikz}
  % Settings for captions, such as sideways captions.
  \usepackage{caption}
  % Symbols for units, like degrees and ohms.
  \usepackage{gensymb}
  % Latin modern fonts - better looking than the defaults.
  \usepackage{lmodern}
  % Allows for columns spanning multiple rows in tables.
  \usepackage{multirow}
  % Better looking tables, including nicer borders.
  \usepackage{booktabs}
  % More math symbols.
  \usepackage{amssymb}
  % More math fonts, like \mathbb.
  \usepackage{amsfonts}
  % More math layouts, equation arrays, etc.
  \usepackage{amsmath}
  % More theorem environments.
  \usepackage{amsthm}
  % More column formats for tables. 
  \usepackage{array}
  % Adjust the sizes of box environments.
  \usepackage{adjustbox}
  % Better looking single quotes in verbatim and minted environments.
  \usepackage{upquote}
  % Better blank space decisions.
  \usepackage{xspace}
  % Better looking tikz trees.
  \usepackage{forest}
  % URLs.
  \usepackage{hyperref}
  % Plotting.
  \usepackage{pgfplots}
  % Page styling.

  \usepackage{fancyhdr}
  % Calculates the number of pages.
  \usepackage{lastpage}
  % Styling the abstract.
  \usepackage{abstract}

  % Titling
  \usepackage{titling}
  
  % Various tikz libraries.
  % For drawing mind maps.
  \usetikzlibrary{mindmap}
  % For adding shadows.
  \usetikzlibrary{shadows}
  % Extra arrows tips.
  \usetikzlibrary{arrows.meta}
  % Old arrows.
  \usetikzlibrary{arrows}
  % Automata.
  \usetikzlibrary{automata}
  % For more positioning options.
  \usetikzlibrary{positioning}
  % Creating chains of nodes on a line.
  \usetikzlibrary{chains}
  % Fitting node to contain set of coordinates.
  \usetikzlibrary{fit}
  % Extra shapes for drawing.
  \usetikzlibrary{shapes}
  % For markings on paths.
  \usetikzlibrary{decorations.markings}
  % For advanced calculations.
  \usetikzlibrary{calc}
  
  % GMIT colours.
  \definecolor{gmitblue}{RGB}{20,134,225}
  \definecolor{gmitred}{RGB}{220,20,60}
  \definecolor{gmitgrey}{RGB}{67,67,67}

  %%%%%% CHANGE
  % Name of the topic.
  \newcommand{\topicname}{Collatz Turing Machine}
  \newcommand{\blm}{\sqcup}
  \newcommand{\bl}{\(\blm\)}
  
  %%%%%%

  % Abstract
  \renewcommand{\abstractname}{}
  \renewcommand{\absnamepos}{empty}

  % Environments
  \newtheorem*{remark}{Remark}
  \newtheorem*{definition}{Definition}

  % Tables
  \newcolumntype{x}[1]{>{\centering\arraybackslash\hspace{0pt}}p{#1}}

  % Bibliography
  \renewcommand{\refname}{\selectfont\normalsize References} 

  \title{\topicname}
  \author{}
  \date{}

  \pgfplotsset{compat=1.16}
  
  \pagestyle{fancy}
  \fancyhf{}
  \renewcommand{\headrulewidth}{0pt}
  \rhead{\url{ian.mcloughlin@gmit.ie}}
  
  \lhead{\today}
 
\setlength{\droptitle}{-40pt}
 %%begin novalidate
\posttitle{\par\end{center}\vspace{-50pt}}
 %%end novalidate
\setlength{\columnsep}{20mm}
\setlength{\columnseprule}{0.1mm}

\setlength{\parindent}{0pt} 
 
\begin{document}
  
  \maketitle
  \thispagestyle{fancy}

  \begin{align*}
      A &= \{ 0, 1 \} \\
      T &= A \ \cup \ \{ , \} \ \cup \  \{ \blm,  X \} \\
      L &= \{1, 10, 11, 100, 101, 110, 111, 1000, 1001, \ldots \} \\
      f(s) &= 1s,s \\
  \end{align*}

  
  \begin{table}[H]
    \centering
    \begin{tabular}{rl}
        \textbf{State} & \textbf{Description} \\
        \midrule
        \( q_0 \) & Append a comma. \\
        \( q_1 \) & Move left to start. \\
        \( q_2 \) & Check if 0 or 1, mark X. \\
        \( q_3 \) & Move right to end, append 0. \\
        \( q_4 \) & Move right to end, append 1. \\
        \( q_5 \) & Move left to X, overwrite 0, back to \(q_2\). \\
        \( q_6 \) & Move left to X, overwrite 1, back to \(q_2\). \\
        \( q_7 \) & Move to start, prepend 1. \\
        \( q_8 \) & Back to start, end. \\
    \end{tabular}
  \end{table}

  \begin{table}
    \centering
    \begin{tabular}{x{1cm}x{1cm}x{1cm}x{1cm}x{1cm}}
      \toprule
      State & Input & Write & Move & Next \\
      \midrule
      % Right to Write a comma at end on input.
      \(q_0\) & \bl &   , & L & \(q_1\) \\
      \(q_0\) &   0 &   0 & R & \(q_0\) \\
      \(q_0\) &   1 &   1 & R & \(q_0\) \\
      \(q_0\) &   , &   , & R & \(q_f\) \\
      \(q_0\) &   X &   X & R & \(q_f\) \\
      \midrule
      % Left to start.
      \(q_1\) & \bl & \bl & R & \(q_2\) \\
      \(q_1\) &   0 &   0 & L & \(q_1\) \\
      \(q_1\) &   1 &   1 & L & \(q_1\) \\
      \(q_1\) &   , &   , & R & \(q_f\) \\
      \(q_1\) &   X &   X & R & \(q_f\) \\
      \midrule
      % Overwrite first symbol with X.
      \(q_2\) & \bl & \bl & R & \(q_f\) \\
      \(q_2\) &   0 &   X & R & \(q_3\) \\
      \(q_2\) &   1 &   X & R & \(q_4\) \\
      \(q_2\) &   , &   , & R & \(q_f\) \\
      \(q_2\) &   X &   X & R & \(q_f\) \\
      \midrule
      % Left to write a 0 at the end.
      \(q_3\) & \bl &   0 & L & \(q_5\) \\
      \(q_3\) &   0 &   0 & R & \(q_3\) \\
      \(q_3\) &   1 &   1 & R & \(q_3\) \\
      \(q_3\) &   , &   , & R & \(q_3\) \\
      \(q_3\) &   X &   X & R & \(q_f\) \\
      \midrule
      % Left to write a 1 at the end.
      \(q_4\) & \bl &   1 & L & \(q_6\) \\
      \(q_4\) &   0 &   0 & R & \(q_4\) \\
      \(q_4\) &   1 &   1 & R & \(q_4\) \\
      \(q_4\) &   , &   , & R & \(q_4\) \\
      \(q_4\) &   X &   X & R & \(q_f\) \\
      \midrule
      % Left to the X, overwrite with 0
      \(q_5\) & \bl & \bl & L & \(q_f\) \\
      \(q_5\) &   0 &   0 & L & \(q_5\) \\
      \(q_5\) &   1 &   1 & L & \(q_5\) \\
      \(q_5\) &   , &   , & L & \(q_5\) \\
      \(q_5\) &   X &   0 & R & \(q_2\) \\
      \midrule
      % Left to the X, overwrite with 1
      \(q_6\) & \bl & \bl & L & \(q_f\) \\
      \(q_6\) &   0 &   0 & L & \(q_6\) \\
      \(q_6\) &   1 &   1 & L & \(q_6\) \\
      \(q_6\) &   , &   , & L & \(q_6\) \\
      \(q_6\) &   X &   1 & R & \(q_2\) \\
      \midrule
      % Left to start, prepend 1.
      \(q_7\) & \bl &   1 & L & \(q_8\) \\
      \(q_7\) &   0 &   0 & L & \(q_7\) \\
      \(q_7\) &   1 &   1 & L & \(q_7\) \\
      \(q_7\) &   , &   , & L & \(q_7\) \\
      \(q_7\) &   X &   X & L & \(q_7\) \\
      \midrule
      % Back to start, accept.
      \(q_7\) & \bl & \bl & R & \(q_a\) \\
      \(q_7\) &   0 &   0 & R & \(q_f\) \\
      \(q_7\) &   1 &   1 & R & \(q_f\) \\
      \(q_7\) &   , &   , & R & \(q_f\) \\
      \(q_7\) &   X &   X & R & \(q_f\) \\
      \bottomrule
    \end{tabular}
    \caption{Turing machine state table}
    \label{table:statetable}
  \end{table}
    
 
  %\bibliographystyle{plain}
  %\bibliography{bibliography}
\end{document}