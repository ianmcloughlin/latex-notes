\documentclass{iansnotes}

  \title{Computability}
  \author{ian.mcloughlin@gmit.ie}
  \date{\today}

\begin{document}

  \maketitle

    \section{Natural numbers}
    \begin{align*}
      \mathbb{N}    &= \{ 1, 2, 3, 4, \ldots \} \\
      \mathbb{N}_0  &= \{ 0, 1, 2, 3, \ldots \} \\
      2\mathbb{N}_0 &= \{ 0, 2, 4, 6, \ldots \} \\
      3\mathbb{N}_0 &= \{ 0, 3, 6, 9, \ldots \} \\
      \mathbb{Z}    &= \{ \ldots, -2, -1, 0, 1, 2, \ldots \} \\
    \end{align*}
    
    \section*{Real numbers}
    
    \begin{center}
    \begin{tikzpicture}
        % https://tex.stackexchange.com/a/148253
        \draw[latex-latex] (-1.5,0) -- (3.5,0) ; %edit here for the axis
        
        \draw[shift={(0,0)},color=black]    (0pt,3pt) -- (0pt,-3pt);
        \draw[shift={(1,0)},color=black]    (0pt,3pt) -- (0pt,-3pt);
        \draw[shift={(1.41,0)},color=black] (0pt,3pt) -- (0pt,-3pt);
        \draw[shift={(3.12,0)},color=black] (0pt,3pt) -- (0pt,-3pt);
        
        \draw[shift={(0,0)},color=black]    (0pt,-3pt) -- (0pt,0pt) node[above] {$0$};
        \draw[shift={(1,0)},color=black]    (0pt,-3pt) -- (0pt,0pt) node[above] {$1$};
        \draw[shift={(1.41,0)},color=black] (0pt,0pt) -- (0pt,-3pt) node[below] {$\sqrt{2}$};
        \draw[shift={(3.14,0)},color=black] (0pt,0pt) -- (0pt,-3pt) node[below] {$\pi$};
        %\draw[very thick] (,0) -- (1.92,0);
    \end{tikzpicture}
    \end{center}

    \noindent Real numbers are difficult to define:
    \begin{enumerate}
        \item Draw an infinite
        straight line.
        \item Mark two distinct points.
        \item Call the left-most 0, the right-most 1.
        \item One unit is the length between 0 and 1.
        \item Mark any point on the line and measure its length $x$ in units from 0.
    \end{enumerate}
    Then $x$ is a real number.
    The set $\mathbb{R}$ is the set of all such $x$'s.
  
    \section*{Bijections}
    
    \[ f: \mathbb{N}_0 \rightarrow 2\mathbb{N}_0 : n \rightarrow 2n  \]
    
    \begin{align*}
        f(0) &= 0 \\
        f(1) &= 2 \\
        f(2) &= 4 \\
        f(3) &= 6 \\
        \vdots \\
    \end{align*}
      
    \noindent
    Every element of $\mathbb{N}_0$ is mapped to a different element of $2\mathbb{N}_0$.
    For instance, $1$ is the only number mapped to $2$.
    
    Also, every element of $2 \mathbb{N}_0$ is mapped to from some element of $\mathbb{N}_0$.
    There's no element of $2 \mathbb{N}_0$ that isn't the image of some element of $\mathbb{N}_0$.
    
    There are lots of functions like $f$ to and from each of the sets $\mathbb{N}, \mathbb{N}_0, \mathbb{Z}$ and so on, but there is none to $\mathbb{R}$.
    
    \section*{Diagonilsation}
    
    Suppose there was a map $f: \mathbb{N}_0 \rightarrow \mathbb{R}$.
    List out each element $n$ of $\mathbb{N}_0$ with $f(n)$ beside it with all of the decimal places displayed.
    For example, suppose the first few are as follows.
    
    \begin{alignat*}{2}
        0 & \quad \rightarrow \quad &   0&.16346234234234 \ldots \\
        1 & \quad \rightarrow \quad &  10&.56775344747474 \ldots \\
        2 & \quad \rightarrow \quad & 214&.99999999999999 \ldots \\
        3 & \quad \rightarrow \quad &  -1&.33333333333333 \ldots \\
        4 & \quad \rightarrow \quad &   0&.00000000000000 \ldots \\
          & \quad \;\; \vdots    \quad &    &                   \\
    \end{alignat*}

    While the list contains all of the natural numbers $\mathbb{N}_0$, we can show that it can't possibly contain all of the real numbers $\mathbb{R}$.
    We'll use Cantor's diagonal argument to construct a real number not in the list.
    
    Start by constructing a new number $0.$ and add decimal places as follows.
    Take the digit in the first decimal place of the first real number in the list.
    The first real number is $0.1634\ldots$ so the first decimal place contains a $1$.
    Add $1$ to this number, giving $2$, and use this as the digit in the first decimal place of our new number: $0.2$.
    
    Next add 1 to the digit in the second decimal place of the second real number.
    That gives 6, so the digit in the second decimal place of our new number is $6$, giving $0.26$ so far.
    Keep going down the list, adding one to each successive decimal place in each successive number.
    Any time you encounter $9$, adding $1$ would give $10$, so replace the $9$ with $0$ instead.
    That happens for the third real number in the list, so our new number becomes $0.260$ at that point.
    
    The new number differs from the first real number in the list in the first decimal place, the second real number in the second decimal place and so on.
    Therefore, the new number is not in the list --- so the list can't contain all real numbers.
    We have to conclude that the real numbers were not and can not be paired up with the natural numbers.
    
    \section*{Computable numbers}
    Cantor's argument has profound implications for what a computer can do.
    Each file on a computer is stored in binary -- a string over the alphabet $\{ 0, 1 \}$.
    The possible files are then $\{ 0 , 1 \}^* = \{ \epsilon, 0, 1, 00, 01, 10, 11, 000, \ldots \}$.
    We can pair $\mathbb{N}_0$ up with $\{ 0 , 1 \}^*$ using that listing: $0 \rightarrow \epsilon$, $1 \rightarrow 0$, $2 \rightarrow 1$, $3 \rightarrow 00$, and so on.
    
    Executables are just files.
    Consider an executable that prints a real number to the screen.
    For instance, we could write a program to print $0.0000\ldots$ to the screen where it just keeps printing $0$'s.
    It's printing the real number $0$.
    For any given real number, is there a program to print it out?
    There must be real numbers for which no such program exists -- because there are less possible programs than real numbers.
    
    Computers can't even print all numbers out.
 
  \end{document}
